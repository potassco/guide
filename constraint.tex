\section{Constraint Programming}\label{sec:constraint}

This section is not yet ready for publishing
and will be included in one of the forthcoming editions of this guide.

Information on constraint programming with \gringo\ (and \clingcon) can be obtained at the following references.

\begin{itemize}
\item \cite{geossc09a,ostsch12a} (\clingcon, based upon \gringo~3.0.92 and \clasp~1.3.10)
\item Video on \gringo's constraint solving capacities \url{http://youtu.be/QrP8KiDl3hw} (see also \url{http://potassco.sourceforge.net/videos.html})
\item Constraint programming in ASP via \aspartame\ \cite{bageinscsotawe13a}; see also \url{http://www.cs.uni-potsdam.de/aspartame}
\end{itemize}

\subsection{ASP modulo CSP solving with \clingcon}
\label{sec:clingcon}

\subsection{Solving CSPs with \aspartame}
\label{sec:aspartame}

\subsection{\gringo's constraint solving capacities}

\gringo\ features some \textcolor{red}{experimental} means for expressing finite linear constraint satisfaction problems within ASP's modeling language and
for subsequently solving them with off-the-shelf ASP solvers such as \clasp.

Within \gringo's input language,
all constraint relations and constraint variables are (\textcolor{red}{currently}) preceded by a dollar symbol, ``\var{\$}''.
In addition, there is a global constraint
\begin{lstlisting}[numbers=none,escapechar=?]
#disjoint{?$\boldsymbol{t}_1$?:?$c_1$?:?$\boldsymbol{L}_1$?;?\,\dots\,?;?$\boldsymbol{t}_n$?:?$c_n$?:?$\boldsymbol{L}_n$?}
\end{lstlisting}
where $\boldsymbol{t}_i$ and $\boldsymbol{L}_i$ are given as in Section~\ref{subsec:gringo:aggregate},
and $c_i$ is a value (over a constraint expression).
%
The idea is that sets of values labeled with the same term(s) must be disjoint.

The overall compilation follows the order encoding \cite{tatakiba09a,bageinscsotawe13a} and
introduces a Boolean variable for each statement `$X\leq k$' where $X$ is a variable over integers and $k$ is an integer bound.

\marginlabel{To compute both answer sets, invoke:\\
  \code{\mbox{~}clingo \attach{examples/queensC.lp}{queensC.lp} \textbackslash\\
         \mbox{~} -c n=30}\\
  or alternatively:\\
  \code{\mbox{~}gringo~\attach{examples/queensC.lp}{queensC.lp} \textbackslash\\
        \mbox{~} -c n=30 | clasp 0}
}
For illustration,
consider the two following encoding of the $n$-queens puzzle.
In both encodings, the first line fixes the domain of the integer variables 
\code{\$queen(1)} to \code{\$queen(n)}.
Line~3 forbids queens on the same columns and the last two lines address queens on the same diagonals.
%
\lstinputlisting{examples/queensC.lp}
\marginlabel{To compute both answer sets, invoke:\\
  \code{\mbox{~}clingo \attach{examples/queensCa.lp}{queensCa.lp} \textbackslash\\
         \mbox{~} -c n=300}\\
  or alternatively:\\
  \code{\mbox{~}gringo~\attach{examples/queensCa.lp}{queensCa.lp} \textbackslash\\
        \mbox{~} -c n=300 | clasp 0}
}
The next encoding uses the global \code{\#disjoint} constraint.
\lstinputlisting{examples/queensCa.lp}

%%% Local Variables: 
%%% mode: latex
%%% TeX-master: "guide"
%%% End: 
