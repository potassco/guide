\begin{abstract}
This document provides an introduction to the Answer Set Programming (ASP) tools
\gringo, \clasp, and \clingo, developed at the University of Potsdam.
The basic idea of ASP is to express a problem in the form of a logic program so that 
its logical models, called \emph{answer sets}, provide the solutions to the original problem.
%
The first tool, \gringo, is a so-called \emph{grounder} translating
user-provided logic programs (with variables) into equivalent propositional logic programs (without variables).
%
The second tool, \clasp, is a so-called \emph{solver} computing
the answer sets of the propositional programs issued by \gringo.
The third tool, \clingo, combines the functionalities of \gringo\ and \clasp,
and additionally integrates the scripting languages \lua\ and \python\ either
through libraries or embedded code.
This guide, for one, aims at enabling ASP novices
to make use of the aforementioned tools.
For another, it provides a reference of the tools' features
that ASP adepts might be tempted to exploit.
\comment{BK: abstract is for ASP insiders\par TS: rewritten slightly, still to much geeky stuff?}
\vfill
\centering
\fbox{
\begin{tabular}{c}
This document includes many illustrative examples.
\\
For convenience,
they can be saved to disk by clicking their file names.
\\
Depending on the viewer, a right or double-click should initiate saving.
\end{tabular}}
\end{abstract}

%%% Local Variables: 
%%% mode: latex
%%% TeX-master: "guide"
%%% End: 
