
\section{Solver Configuration}
\label{sec:configuration}

\textit{This section is not yet ready for publishing
and will be included in one of the forthcoming editions of this guide.}

\clasp{} has more than $80$ performance relevant parameters, partially show in Section~\ref{subsec:opt:clasp}.
Even if only a discrete subset of all possible parameter configurations is considered,
\clasp{} has approximately $10^{59}$ parameter configurations.
In such a huge configuration space, 
it is a tedious and time-consuming task 
to manually determine a well-performing parameter configuration.
Two complementary ways to automatically address this issue for \clasp{} are 
the automatic algorithm selection solver \claspfolio{}~\cite{holisc14a} and 
the automatic algorithm configuration tool \piclasp{}.\marginlabel{
Both tools are written in \python{} 2.7 and require some external packages -- please see README.}

\subsection{Portfolio-Solving with \claspfolio}
\label{sec:claspfolio}

The use-case of \claspfolio{}\footnote{\url{http://potassco.sourceforge.net\#claspfolio}}~\cite{holisc14a} 
is to solve a set of heterogeneous instances, that is,
there is no single well-performing configuration for all instances, 
but a well-performing configuration has to be selected for each individual instance.
Therefore, \claspfolio{} should be used either (i) 
in applications  where you have to solve instances with different characteristics, 
e.g., due to different encodings, different sizes or changing constraints 
(see Example~\ref{subsec:example:claspfolio})
or (ii) to get a first flavor of a well-performing configuration for your (homogeneous) application (see Example~\ref{subsec:example:claspfolio_oralce}). 

The basic idea of \claspfolio{} consists of using numerical characteristics of instances, so-called instance features, computed by \claspre{}\footnote{\url{http://www.cs.uni-potsdam.de/claspre/}}
to select a well-performing configuration from a given set of pre-selected configurations
for given a grounded logic program by using machine learning techniques.
\begin{example}\label{subsec:example:claspfolio}
Using the $n$-queens puzzle from Example~\ref{ex:csp:queens1}, 
we compare the performance of the default configuration of \clasp{} 
and \claspfolio{}\comment{ML: with clasp2.1, this would be a nice example, but I have to wait for the new claspfolio release to compare against clasp 3.1}
\begin{lstlisting}[numbers=none]
$gringo examples/queensC.lp -c n=60 | clasp
[...]
Time         : XXXs

$gringo examples/queensC.lp -c n=60 | \
   python ./src/claspfolio.py -I -
[...]
Time         : XXXs
\end{lstlisting}

\end{example}

\begin{example}\label{subsec:example:claspfolio_oralce}
Another way to use \claspfolio{} is to select a configuration for a given set of instances.
In such a setting, \claspfolio{} will score each configuration on each instance 
and averages over the scores of each configuration.
\begin{lstlisting}[numbers=none]
$ python ./src/claspfolio.py --oracle_dir <INSTANCE_DIR>
% [...]
%  >>> Algorithm Scores <<<
% 
% 1-th ranked solver: 	 <CONFIGURATION NAME>
% Call: 		 <CMD CALL>
% Score: 		 <SCORE>
% ...
\end{lstlisting}

\claspfolio{} lists all configuration sorted by its performance score -- starting with predicted best-performing configuration.
Please note that \claspfolio{} minimizes \texttt{<SCORE>}.
\end{example}

\note{
\claspfolio{} is trained for a runtime cutoff of $600$ seconds. 
It will most likely perform well for smaller runtime cutoffs 
but the performance of \claspfolio{} could get worse by using a larger runtime cutoff.
}

\note{
\claspfolio{} is trained only on decision problems.
Therefore, \claspfolio{} does not cover enumeration and optimization related parameters in its selected configurations.
}

\subsection{Benchmark-oriented \clasp\ Configuration with \piclasp}
\label{sec:piclasp}

In contrast to \claspfolio{}, 
\piclasp{}\footnote{\url{http://www.cs.uni-potsdam.de/piclasp}} looks for a well-performing parameter configuration 
in the complete parameter configuration space of \clasp{}.
To this end, \piclasp{} optimizes \clasp{}'s configuration with the automatic algorithm configuration framework \smac{}~\cite{huhole11b}.
In the process of determining a configuration,
\piclasp{} has to assess the performances of different \clasp{} configurations on different instances. 
Therefore, \piclasp{} needs a lot more computational ressources in comparison to \claspfolio{}
but has the advantage by adapting \clasp{} even better to your own application.
 
\piclasp{} has two required parameters:
%
\begin{description}
  \item[\code{--instances,-I}] a set of grounded instances on which the performance of \clasp{} will be optimized.
  \item[\code{--cutoff,-c}] defines the runtime cutoff of each run of \clasp. 
  		We recommend that \clasp{}'s default configuration should solve at least $50\%$ of the given instances with this cutoff.
  		The runtime of \piclasp{} (i.e., the configuration budget) will be approx. $200$ times this runtime cutoff 
  		to determine a well-performing configuration of \clasp{}.
\end{description} 

To install all required packages of \piclasp{}, please run \code{bash install.sh}. 
It will locally install \clasp{}, \smac{}, \sysfont{runsolver} and \claspre{}.

\begin{example}
For illustration, consider to use \piclasp{} to determine a well-performing configuration for the $n$-queens puzzle, 
you have to provide a directory with the grounded instances, e.g., 
\code{gringo examples/queensC.lp -c n=60 > <INSTANCE\_DIR>/queens\_60.gr}

Running \piclasp{} on this one instance:
\begin{lstlisting}[numbers=none]
$python piclasp.py -c 10 -I <INSTANCE_DIR>
Found 1 instances
[...]
Result of piclasp:
Performance: 1.217000
--learn-explicit --no-gamma --sat-prepro=0 --trans-ext=all
--backprop --eq=2 --sign-def=1 --del-max=1601416487
--strengthen=local,0 --loops=no --init-watches=2
--heuristic=None --score-other=0 --del-cfl=F,198
--restarts=L,705,4 --partial-check=50 --del-estimate=2
--del-grow=1.4787,36.1154 --update-act --del-glue=6,0
--update-lbd=1 --reverse-arcs=0 --deletion=basic,38,1
--rand-freq=0.05 --otfs=2 --del-on-restart=29 
--contraction=6 --counter-restarts=22 
--del-init=25.0434,57,20840 --local-restarts 
--lookahead=no --save-progress=2
--counter-bump=1435 --sign-fix
\end{lstlisting}

Comparing the performance of \clasp{}'s default configuration
and the configuration determined by \piclasp{} shows a two-fold speedup.

\begin{lstlisting}[numbers=none]
$ cat queens_60.gr | clasp
[...]
Time         : 2.780s

$ cat queens_60.gr | clasp <PICLASP CONFIGURATION>
[...]
Time         : 1.251s
\end{lstlisting}

Interestingly, the configuration determined by \piclasp{} disabled the variable selection heuristic (\code{--heuristic=None})
which a human will probably never do.

\end{example}

\note{
To improve the performance of \piclasp, 
we recommend to run \piclasp{} with at least $10$ independent \smac{} runs (option \code{--repetition,-R}).
More \smac{} runs or a larger configuration budget (option \code{--budget,-B}) should always lead to better results.   
}

\note{
Algorithm configuration and hence also \piclasp{} works especially well on homogeneous instance sets (e.g.,~\cite{gejokaobsascsc13a}),
that is, there is one configuration that performs well on all given instances.
On heterogeneous instance sets, \piclasp{} will most likely need a lot more \smac{} runs and a larger configuration budget,
and it will still find only configuration with small performance improvements, 
since \clasp{}'s default configuration is already optimized to have a robust performance on a large variety of instances.
}

\note{
Using \piclasp{}, the performance of \clasp{} on the given instance set will improve.
However ultimately, the performance of \clasp{} should improve on new (unseen) instances.
Therefore, we strongly recommend to use another (disjoint) set of instances to assess the performance of the configured \clasp{}.
}

\note{
We recommend that \piclasp{} optimizes the performance of \clasp{} on at least $100$ instances.
On smaller instance sets, the determined configuration may will not perform well on new (unseen) instances.}


%%% Local Variables: 
%%% mode: latex
%%% TeX-master: "guide"
%%% End: 
