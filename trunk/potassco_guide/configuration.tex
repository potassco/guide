
\section{Solver Configuration}
\label{sec:configuration}

\clasp\ has more than $80$ performance relevant parameters, some of which are shown in Section~\ref{subsec:opt:clasp}.
Even if only a discrete subset of all possible parameter settings is considered,
this amounts to approximately $10^{59}$ configurations.
In such a huge configuration space, 
it is a tedious and time-consuming task 
to manually determine a well-performing configuration.
Two complementary ways to automatically address this issue for \clasp\ are 
the automatic \REPc{algorithm}{configuration}{T2M: I guess that's better in teh context of the guide, or?} selection solver \claspfolio~\cite{holisc14a} and 
the automatic \DEL{algorithm} configuration tool \piclasp.%
\marginlabel{Both tools are written in \python~2.7 and require some external packages -- please see README.}
\comment{T2M: Please include README}

\subsection{Portfolio-Solving with \claspfolio}
\label{sec:claspfolio}

The targeted use-case of \claspfolio\ is to solve a set of heterogeneous problem instances.
In such a case,
there is no single well-performing configuration for all instances
but a well-performing configuration has to be selected for each individual instance.
Therefore, \claspfolio\ should be used either 
in scenarios involving instances with different characteristics, 
e.g., due to different encodings, different sizes or changing constraints 
(see Example~\ref{subsec:example:claspfolio}),
or 
simply to get a first impression of a well-performing configuration of a (homogeneous) benchmark set (see Example~\ref{subsec:example:claspfolio_oracle}). 

The basic idea of \claspfolio\ consists of using numerical characteristics of instances
to select a well-performing configuration from a given set of pre-selected configurations
by using machine learning techniques
in order to solve a given (ground) logic program at hand.
These so-called instance features are computed by \claspre. % footnote{\url{http://www.cs.uni-potsdam.de/claspre/}}
%
\begin{example}\label{subsec:example:claspfolio}
Using the $n$-queens puzzle from Example~\ref{ex:csp:queens1}, 
we compare the performance of the default configuration of \clasp\ 
and \claspfolio{}\comment{ML: with clasp2.1, this would be a nice example, but I have to wait for the new claspfolio release to compare against clasp 3.1}
\comment{T2M: So far, we tried to use example environments only when using a marginpar with downloadable files }
\comment{T2M: Why use claspfolio as a script and not standalone??}
\begin{lstlisting}[numbers=none]
$ gringo examples/queensC.lp -c n=60 | clasp
[...]
Time         : XXXs

$ gringo examples/queensC.lp -c n=60 | \
   python ./src/claspfolio.py -I -
[...]
Time         : XXXs
\end{lstlisting}

\end{example}

\begin{example}\label{subsec:example:claspfolio_oracle}
Another way to use \claspfolio\ is to select a configuration for a given set of instances.
In such a setting, \claspfolio\ scores each configuration on each instance 
and averages over the scores of each configuration.
\comment{T2M: Da kommt noch was, oder?}
\begin{lstlisting}[numbers=none]
$ python ./src/claspfolio.py --oracle_dir <INSTANCE_DIR>
% [...]
%  >>> Algorithm Scores <<<
% 
% 1-th ranked solver: 	 <CONFIGURATION NAME>
% Call: 		 <CMD CALL>
% Score: 		 <SCORE>
% ...
\end{lstlisting}
%
\claspfolio\ lists all configuration sorted by its performance score --- starting with predicted best-performing configuration.
Please note that \claspfolio\ minimizes \texttt{<SCORE>}.
\end{example}

\begin{note}
\claspfolio\ is trained for a runtime cutoff of $600$ seconds. 
It will most likely perform well for smaller runtime cutoffs 
but performance could get worse with larger runtime cutoffs.
\end{note}

\begin{note}
\claspfolio\ is trained only on decision problems.
Therefore, \claspfolio\ does not cover enumeration and optimization related parameters in its selected configurations.
\end{note}

\comment{T2M: Any discussion of options?}

\subsection{Benchmark-oriented \clasp\ Configuration with \piclasp}
\label{sec:piclasp}

In contrast to \claspfolio, 
\piclasp\footnote{\url{http://www.cs.uni-potsdam.de/piclasp}} looks for a well-performing parameter configuration 
in the complete parameter configuration space of \clasp.
To this end, \piclasp\ optimizes \clasp's configuration with the automatic algorithm configuration framework \smac~\cite{huhole11b}.
In the process of determining a configuration,
\piclasp\ has to assess the performances of different \clasp\ configurations on different instances. 
Therefore, \piclasp\ needs a lot more computational ressources in comparison to \claspfolio\
but has the advantage by adapting \clasp\ even better to your own application.
 
\piclasp\ has two required parameters:
%
\begin{description}
  \item[\code{--instances,-I}] a set of grounded instances on which the performance of \clasp\ will be optimized.
  \item[\code{--cutoff,-c}] defines the runtime cutoff of each run of \clasp. 
  		We recommend that \clasp's default configuration should solve at least $50\%$ of the given instances with this cutoff.
  		The runtime of \piclasp\ (i.e., the configuration budget) will be approx. $200$ times this runtime cutoff 
  		to determine a well-performing configuration of \clasp.
\end{description} 

To install all required packages of \piclasp, please run \code{bash install.sh}. 
It will locally install \clasp, \smac, \sysfont{runsolver} and \claspre.

\begin{example}
For illustration, consider to use \piclasp\ to determine a well-performing configuration for the $n$-queens puzzle, 
you have to provide a directory with the grounded instances, e.g., 
\code{gringo examples/queensC.lp -c n=60 > <INSTANCE\_DIR>/queens\_60.gr}

Running \piclasp\ on this one instance:
\begin{lstlisting}[numbers=none]
$python piclasp.py -c 10 -I <INSTANCE_DIR>
Found 1 instances
[...]
Result of piclasp:
Performance: 1.217000
--learn-explicit --no-gamma --sat-prepro=0 --trans-ext=all
--backprop --eq=2 --sign-def=1 --del-max=1601416487
--strengthen=local,0 --loops=no --init-watches=2
--heuristic=None --score-other=0 --del-cfl=F,198
--restarts=L,705,4 --partial-check=50 --del-estimate=2
--del-grow=1.4787,36.1154 --update-act --del-glue=6,0
--update-lbd=1 --reverse-arcs=0 --deletion=basic,38,1
--rand-freq=0.05 --otfs=2 --del-on-restart=29 
--contraction=6 --counter-restarts=22 
--del-init=25.0434,57,20840 --local-restarts 
--lookahead=no --save-progress=2
--counter-bump=1435 --sign-fix
\end{lstlisting}

Comparing the performance of \clasp's default configuration
and the configuration determined by \piclasp\ shows a two-fold speedup.

\begin{lstlisting}[numbers=none]
$ cat queens_60.gr | clasp
[...]
Time         : 2.780s

$ cat queens_60.gr | clasp <PICLASP CONFIGURATION>
[...]
Time         : 1.251s
\end{lstlisting}

Interestingly, the configuration determined by \piclasp\ disabled the variable selection heuristic (\code{--heuristic=None})
which a human will probably never do.

\end{example}

\begin{note}
To improve the performance of \piclasp, 
we recommend to run \piclasp\ with at least $10$ independent \smac\ runs (option \code{--repetition,-R}).
More \smac\ runs or a larger configuration budget (option \code{--budget,-B}) should always lead to better results.   
\end{note}

\begin{note}
Algorithm configuration and hence also \piclasp\ works especially well on homogeneous instance sets (e.g.,~\cite{gejokaobsascsc13a}),
that is, there is one configuration that performs well on all given instances.
On heterogeneous instance sets, \piclasp\ will most likely need a lot more \smac\ runs and a larger configuration budget,
and it will still find only configuration with small performance improvements, 
since \clasp's default configuration is already optimized to have a robust performance on a large variety of instances.
\end{note}

\begin{note}
Using \piclasp, the performance of \clasp\ on the given instance set will improve.
However ultimately, the performance of \clasp\ should improve on new (unseen) instances.
Therefore, we strongly recommend to use another (disjoint) set of instances to assess the performance of the configured \clasp.
\end{note}

\begin{note}
We recommend that \piclasp\ optimizes the performance of \clasp\ on at least $100$ instances.
On smaller instance sets, the determined configuration may will not perform well on new (unseen) instances.
\end{note}


%%% Local Variables: 
%%% mode: latex
%%% TeX-master: "guide"
%%% End: 
