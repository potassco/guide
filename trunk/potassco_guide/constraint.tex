\section{Constraint Programming}\label{sec:constraint}

\subsection{ASP modulo CSP solving with \clingcon}
\label{sec:clingcon}

This section is not yet ready for publishing
and will be included in one of the forthcoming editions of this guide.

Information on constraint programming with \clingcon\ can be obtained at the following references.

\begin{itemize}
\item \url{http://potassco.sourceforge.net#clingcon}
\item \cite{geossc09a,ostsch12a} (\clingcon, based upon \gringo~3.0.92 and \clasp~1.3.10)
\end{itemize}

\subsection{Solving CSPs with \aspartame}
\label{sec:aspartame}

This section is not yet ready for publishing
and will be included in one of the forthcoming editions of this guide.

Information on constraint programming with \aspartame\ can be obtained at the following references.

\begin{itemize}
\item \url{http://potassco.sourceforge.net/labs.html}
\item \url{http://www.cs.uni-potsdam.de/aspartame}
\item \cite{bageinscsotawe13a} describes \aspartame\ and how it allows for solving finite linear CSPs in ASP
\end{itemize}

\subsection{Constraint Programming with \gringo}

Grounder \gringo\ features some \textcolor{red}{experimental} means for expressing finite linear constraint satisfaction problems within ASP's modeling language.
The linear constraints are compiled into normal rules following the order encoding \cite{tatakiba09a,bageinscsotawe13a}.
Hence, off-the-shelf ASP solvers like \clasp\ can be used to solve such problems.

CSP constraints in gringo are build over \emph{constraint terms}, which have form:
\begin{lstlisting}[numbers=none,escapechar=?]
?$c_1$? $* $?$v_1$? $+ ?$\cdots$? $+ ?$c_n$? $* $?$v_n$?
\end{lstlisting}
where $n>0$, and each $c_i$ (integer factor) and $v_i$ (name of a constraint variable) are terms.
If a factor is one, then the `\code{$c_i$ \$*}' part can be omitted.
Similarly, it is possible to just add a factor in which case the `\code{\$* $v_i$}' part can be omitted.

\emph{Linear constraints} in \gringo\ are syntactically similar to built-in comparison predicates (cf. Section~\ref{subsec:gringo:comp})
but relation symbols have to be preceded with a \code{\$} symbol:
\begin{lstlisting}[numbers=none,escapechar=?]
?$t_0$? $?$\prec_1$? ?$\cdots$? $?$\prec_{n}$? ?$t_n$?
\end{lstlisting}
where $n>0$, each $\prec_i$ is a comparison predicate, and each $t_i$ is a constraint term.

In addition, there is the global \emph{disjoint constraint}:
\begin{lstlisting}[numbers=none,escapechar=?]
#disjoint{?$\boldsymbol{t}_1$?:?$c_1$?:?$\boldsymbol{L}_1$?;?\,\dots\,?;?$\boldsymbol{t}_n$?:?$c_n$?:?$\boldsymbol{L}_n$?}
\end{lstlisting}
where $\boldsymbol{t}_i$ and $\boldsymbol{L}_i$ are given as in Section~\ref{subsec:gringo:aggregate},
and each $c_i$ is a constraint term.
%
The idea is that sets of values labeled with the same term(s) must be disjoint.

\begin{example}\label{ex:csp:queens1}
\marginlabel{To compute both answer sets, invoke:\\
  \code{\mbox{~}clingo \attach{examples/queensC.lp}{queensC.lp} \textbackslash\\
         \mbox{~} -c n=30}\\
  or alternatively:\\
  \code{\mbox{~}gringo~\attach{examples/queensC.lp}{queensC.lp} \textbackslash\\
        \mbox{~} -c n=30 | clasp 0}
}
For illustration,
consider the following encoding of the $n$-queens puzzle:
\lstinputlisting{examples/queensC.lp}

The first line fixes the domain of the integer variables
\code{\$queen(1)} to \code{\$queen(n)}.
Line~3 forbids queens on the same columns and the last two lines address queens on the same diagonals.
\end{example}

\begin{example}
The next encoding uses the global \code{\#disjoint} constraint:
%
\marginlabel{To compute both answer sets, invoke:\\
  \code{\mbox{~}clingo \attach{examples/queensCa.lp}{queensCa.lp} \textbackslash\\
         \mbox{~} -c n=300}\\
  or alternatively:\\
  \code{\mbox{~}gringo~\attach{examples/queensCa.lp}{queensCa.lp} \textbackslash\\
        \mbox{~} -c n=300 | clasp 0}
}
%
\lstinputlisting{examples/queensCa.lp}
\end{example}

\begin{note}
The \textcolor{red}{current} implementation of constraints in \gringo\ requires 
that all constraint variables appearing in a program must have finite domains inferable from the grounded program.
Hence, rules like in Line~1 of Example~\ref{ex:csp:queens1} fixing the domain of a constraint variable have to be added for each constraint variable.
\end{note}

%%% Local Variables: 
%%% mode: latex
%%% TeX-master: "guide"
%%% End: 
