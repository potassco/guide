
\section{Heuristic-driven Solving}
\label{sec:heuristic}
\comment{T2J: Add index statements}

\comment{C: What kind of heuristic is that? What does this do? (an introduction seems missing to me)}
\clasp\ and \clingo\ provide means for incorporating domain-specific heuristics into ASP solving.
This allows for modifying the heuristic of the solver from within a logic program or from the command line.
A formal description can be found in \cite{gekaotroscwa13a}.
\comment{C: What framework? What kind of framework is implemented as a heuristic?}
The framework is implemented as a new heuristic, named \code{Domain},
that extends the \code{Vsids} heuristics of \clasp\ 
and can be activated using option \code{--heuristic=Domain}
(cf.\ Section~\ref{subsec:clasp:tune}).
In what follows,
we first describe how to modify the solver's heuristic from within a logic program, 
and then we explain how to apply modifications from the command line.

\subsection{Heuristic Programming}

Heuristic information is represented within a logic program my means of the dedicated predicate \code{\_heuristic}.
Different types of heuristic information can be controlled with the modifiers 
\code{sign}, \code{level}, \code{true}, \code{false}, \code{init} and \code{factor}.
We introduce them below step by step.

\subsubsection{Heuristic modifier \code{sign}}

The modifier \code{sign} allows for controlling the truth value assigned to variables subject to a choice within the solver.

The \code{Domain} heuristic associates with each atom an integer \code{sign} value, which by default is $0$.
When deciding which truth value to assign to an atom during a choice, the atom is assigned to true, if its \code{sign} value is greater than $0$. 
If the \code{sign} value is less than 0, it is assigned to false.
And if it is $0$, the sign is determined by the default sign heuristic.

In order to associate a positive sign with atom \code{a}, we can use the heuristic atom \code{\_heuristic(a,sign,1)}. 
This associates a positive \code{sign} with \code{a} and 
tells the solver that upon deciding the atom \code{a}, it should be set to true.
\begin{example}
\label{example:psign}
Consider the following program:
\lstinputlisting[numbers=none]{examples/psign.lp}
\marginlabel{%
  To inspect the output, invoke:\\
  \code{\mbox{~}clingo \attach{examples/psign.lp}{psign.lp} \textbackslash\\\mbox{~}--heuristic=Domain}\\
  or alternatively:\\
  \code{\mbox{~}gringo \attach{examples/psign.lp}{psign.lp} \textbackslash\\\mbox{~}| clasp \textbackslash\\\mbox{~}--heuristic=Domain}\\
}
At the start of the search, 
the solver propagates the heuristic atom 
and updates its heuristic knowledge about atom \code{a}.
Then, it has to decide on \code{a}, making it either true or false.
Following the current heuristic knowledge, 
the solver makes \code{a} true and returns the answer set \code{\{\_heuristic(a,sign,1),a\}}.
\end{example}

\begin{note}
The result would be the same if in the heuristic fact we used instead of $1$ any positive integer.
\end{note}

\begin{example}
\label{example:nsign}
In the next program,
the \code{\_heuristic} fact gives \code{a} a negative sign 
and thus asserts that when deciding upon \code{a} it should be set to false:
\lstinputlisting[numbers=none]{examples/nsign.lp}
\marginlabel{%
  To inspect the output, invoke:\\
  \code{\mbox{~}clingo \attach{examples/nsign.lp}{nsign.lp} \textbackslash\\\mbox{~}--heuristic=Domain}\\
  or alternatively:\\
  \code{\mbox{~}gringo \attach{examples/nsign.lp}{nsign.lp} \textbackslash\\\mbox{~}| clasp \textbackslash\\\mbox{~}--heuristic=Domain}\\
}
As above,
the solver starts with propagating the heuristic fact,
then
updates its heuristic knowledge,
decides on atom \code{a} making it false, and 
finally 
returns the answer set \mbox{\{ \code{\_heuristic(a,sign,-1)}\}}.
\end{example}

These two examples illustrate how the heuristic atoms allow for modifying the decisions of the solver,
leading to either finding first the answer set with \code{a} or the one without it.
However, as long as heuristic atoms appear only in the head of rules,
the program's overall answer sets remain the same (modulo heuristic atoms).
For example, if we ask for all answer sets in Example \ref{example:psign},
we obtain one without \code{a} and one with \code{a},
and the same happens with Example \ref{example:nsign}, 
although in this case the answer sets are computed in opposite order.

\subsubsection{Showing heuristic information}
\comment{C: Expected section for next modifier. Please consider changing the structure}

\comment{C: Is there a reason for this?}
For the heuristic atoms to take effect, they as well as the atoms to which they refer to
must be shown in the logic program.\footnote{Note that in \gringo\ all atoms are shown, 
  unless \code{\#show} statements appear in the program.}
For example, if we add to Example \ref{example:nsign} the line:
\begin{lstlisting}[numbers=none]
#show a/0.
\end{lstlisting}
then the heuristic atom is not shown, 
and the solver operates as if it would normally do with the \code{Vsids} heuristic. 
The same would happen if instead we only added:
\begin{lstlisting}[numbers=none]
#show _heuristic/3.
\end{lstlisting}
because then the atom \code{a}, appearing in the heuristic atom, would not be shown.

\begin{note}
Printing the heuristic atoms in the output of \clingo\ or \clasp\ may often be a burden.
To overcome this issue,
the command line option \code{--out-hide-aux} allows us to suppress printing atoms starting with underscore `\code{\_}'.
\comment{C: Does hiding mean that they will not take effect?}
\end{note}

\subsubsection{Heuristic modifier \code{level}}

The \code{Domain} heuristic assigns to each atom a \code{level}, and it decides first upon atoms of the highest level.
The default value for each atom is \code{0},  and both positive and negative integers are valid.
\begin{example} 
\label{example:level}
In this example,  
level \code{10} is assigned to atom \code{a}:
\lstinputlisting[numbers=none]{examples/level.lp}
\marginlabel{%
  To inspect the output, invoke:\\
  \code{\mbox{~}clingo \attach{examples/level.lp}{level.lp} \textbackslash \\\mbox{~}--heuristic=Domain} \\
  or alternatively:\\
  \code{\mbox{~}gringo \attach{examples/level.lp}{level.lp} \textbackslash \\\mbox{~}| clasp \textbackslash\\\mbox{~}--heuristic=Domain}\\
}
The first obtained answer set contains \code{a} along with all three heuristic atoms.
The solver propagates the heuristic facts, and given that the level of \code{a} is greater than that of \code{b},
it decides first on \code{a} (with positive sign) and then \code{b} is propagated to false.
If we added the fact \code{\_heuristic(b,level,20)}, 
we would first obtain the answer set containing \code{b} instead of \code{a}.
This would also be the case if we used \code{\_heuristic(a,level,-10)} instead of \code{\_heuristic(a,level,10)}.
\end{example}

\begin{note}
The \code{Domain} heuristic is an extension of the \code{Vsids} heuristic, 
so when there are many unassigned atoms with the highest level,
the heuristic decides, among them, on the one with the highest \code{Vsids} score.
\end{note}

\subsubsection{Dynamic heuristic modifications}

Heuristic atoms can be used as any other atom within logic programs,
and they only affect the heuristic of the solver when they are true.
\begin{example}
\label{example:dynamic}
In the next program, the heuristic atoms for \code{c} depend on \code{b}:
\lstinputlisting[numbers=none]{examples/dynamic.lp}
\marginlabel{%
  To inspect the output, invoke:\\
  \code{\mbox{~}clingo \attach{examples/dynamic.lp}{dynamic.lp} \textbackslash\\ \mbox{~}--heuristic=Domain}\\
  or alternatively:\\
  \code{\mbox{~}gringo \attach{examples/dynamic.lp}{dynamic.lp} \textbackslash\\ \mbox{~}| clasp \textbackslash\\ \mbox{~}--heuristic=Domain}\\
}
The first obtained answer set contains \code{a} and \code{\_heuristic(c,sign,-1)} along with the three heuristic atoms given as facts.
At first, the solver proceeds as in Example \ref{example:level}.
Then, after propagating \code{b} to false,  
the heuristic fact \code{\_heuristic(c,sign,-1)} is propagated.
So, when deciding upon \code{c}, it gets assigned to false.  
If we added the fact \code{\_heuristic(b,level,20)},
the obtained first answer set would contain \code{b} and \code{c} instead of \code{a}.
\end{example}

\subsubsection{Heuristic modifiers \code{true} and  \code{false}}

The modifiers \code{true} and \code{false} allow us to refer at the same time to the \code{level} and the \code{sign} of an atom.
Internally,  for the \code{true} and \code{false} heuristic atoms,
the solver defines new \code{level} and \code{sign} heuristic atoms following these rules:
\begin{lstlisting}[numbers=none]
_heuristic(X,level,Y) :- _heuristic(X,true,Y).
_heuristic(X,sign,1)  :- _heuristic(X,true,Y).
_heuristic(X,level,Y) :- _heuristic(X,false,Y).
_heuristic(X,sign,-1) :- _heuristic(X,false,Y).
\end{lstlisting}
For instance, the program of Example \ref{example:dynamic}
can be rewritten as:
\begin{lstlisting}[numbers=none]
_heuristic(b,sign,1).
_heuristic(a,true,10).
{a,b}.
:- a, b.
{c}.
_heuristic(c,sign,1)  :- b.
_heuristic(c,sign,-1) :- not b.
\end{lstlisting}
In this case, the fact \code{\_heuristic(a,true,10)} stands for the previous
facts \code{\_heuristic(a,level,10)} and \code{heuristic(a,sign,1)}.

\subsubsection{Priorities among heuristic modifications}

The \code{Domain} heuristic allows for representing priorities between different heuristic atoms that refer to the same atom.
The priority is optionally represented by a positive integer as a fourth argument.
The higher the integer, the higher the priority of the heuristic atom.
For example, \code{\_heuristic(c,sign,1,10)} and \code{\_heuristic(c,sign,-1,20)} are valid heuristic atoms.
If both are true, then the sign assigned to \code{c} is \code{-1} (because priority \code{20} overrules \code{10}).  

\begin{example}
\label{example:priority}
Consider the following program:
\lstinputlisting[numbers=none]{examples/priority.lp}
\marginlabel{%
  To inspect the output, invoke:\\
  \code{\mbox{~}clingo \attach{examples/priority.lp}{priority.lp} \textbackslash\\ \mbox{~}--heuristic=Domain}\\
  or alternatively:\\
  \code{\mbox{~}gringo \attach{examples/priority.lp}{priority.lp} \textbackslash\\ \mbox{~}| clasp \textbackslash\\ \mbox{~}--heuristic=Domain}\\
}
The first obtained answer set contains \code{a} and \code{\_heuristic(c,sign,-1,20)} along with the three heuristic atoms given as facts.
First the solver proceeds as in Example \ref{example:level}. 
Then, after setting \code{b} to false by propagation, 
the heuristic atom \code{\_heuristic(c,sign,-1,20)} is propagated.
Given that priority \code{20} is greater than \code{10},
the \code{sign} value of atom \code{c} is \code{-1}.
So, when deciding upon \code{c}, it is assigned to false.
If we added the fact \code{\_heuristic(c,sign,1,30)}, 
the first answer set would also contain atom \code{c}.
\end{example}

\begin{note}
Whenever we only use ternary \code{\_heuristic} atoms, 
the assigned priority is the absolute value of the modifiers' values.
For example, 
if both \code{\_heuristic(c,level,-10)} and \code{\_heuristic(c,level,5)} are true,
the level of \code{c} is \code{-10} because $|\code{-10}|>|\code{5}|$.
\end{note}

\subsubsection{Heuristic modifiers \code{init} and \code{factor}}

The modifiers \code{init} and \code{factor} allow for modifying the scores assigned to atoms by the underlying \code{Vsids} heuristic.
Unlike the \code{level} modifier, 
\code{init} and \code{factor} allow us to bias the search without establishing a strict ranking among the atoms.

With \code{init}, we can add a value to the initial heuristic score of an atom.
For example, if \code{\_heuristic(a,init,2)} is true, then a value of \code{2} is added to
the initial score that the heuristic assigns to atom \code{a}.
Note that as the search proceeds, the initial score of an atom decays, 
so \code{init} only affects the beginning of the search.

To bias the whole search, we can use the \code{factor} modifier
that multiplies the heuristic score of an atom by a given value.
For example, if \code{\_heuristic(a,factor,2)} is true, 
then the heuristic score for atom \code{a} is multiplied by \code{2}.

\subsubsection{Monitoring \code{domain} choices}

The \code{Domain} heuristic extends \clasp\ and \clingo's search statistics produced with command line option \code{--stats}.
After `\code{Domain:}', it prints how many decisions where made on atoms appearing inside \code{\_heuristic} atoms.
%
For instance, the statistics obtained in Example \ref{example:dynamic} read as follows.
\begin{lstlisting}[numbers=none]
...
Models      : 1+
Time        : 0.000s (Solving: 0.00s ...)
CPU Time    : 0.000s
Choices     : 2      (Domain: 2)
Conflicts   : 0
Restarts    : 0
...
\end{lstlisting}
The line about \code{Choices} tells us that two decisions were made 
and that both where made on atoms contained in \code{\_heuristic} atoms.

\begin{example}
\label{example:blocks:world:heuristic}
\comment{JR: This should be made coherent with the Incremental Solving Section, 
             where Blocks World also appears.}
We now apply the \code{Domain} heuristic to a Blocks World planning planning.  
We begin with a basic encoding of the problem:
\lstinputlisting[numbers=none]{examples/bw.lp}
and one instance:
\lstinputlisting[numbers=none]{examples/bw_ins.lp}
%
%If we execute:
%\begin{lstlisting}[numbers=none]
%$ gringo bw.lp ins | clasp \end{lstlisting}
%\clasp\ will run with \code{vsids} heuristic. Now we can try to improve the performance of the system programming the heuristic.
In this encoding, once all the values for predicate \code{move} are given,
the values of \code{on} and \code{non} are determined and may be propagated by the solver.
This suggests that deciding only on \code{move} may be a good strategy.
We can do that with the \code{Domain} heuristic 
adding the following heuristic rule:

%\begin{lstlisting}[numbers=none]
\begin{lstlisting}[basicstyle=\small\ttfamily,numbers=none]
_heuristic(move(B,L,T),level,1) :- block(B),location(L),time(T).
\end{lstlisting}
Given that the level of \code{move} is higher,  the solver 
decides first on atoms of that predicate,
and the values of the other predicates are propagated.

We may prefer to soften the heuristic modification to simply bias the search  towards the \code{move} predicate,
without establishing a strict preference towards it.  
For that, we can use, for example, the rule
%\begin{lstlisting}[numbers=none]
\begin{lstlisting}[basicstyle=\small\ttfamily,numbers=none]
_heuristic(move(B,L,T),init,  2) :- block(B),location(L),time(T).
\end{lstlisting} 
or
%\begin{lstlisting}[numbers=none]
\begin{lstlisting}[basicstyle=\small\ttfamily,numbers=none]
_heuristic(move(B,L,T),factor,2) :- block(B),location(L),time(T).
\end{lstlisting}
or the combination of both.  
The first rule adds \code{2} to the initial score of \code{move} atoms,
while the second multiplies the heuristic score of \code{move} by \code{2}.

Whenever we decide on making a \code{move} atom true,  
the other \code{move} atoms for the same \code{time} are determined to be false,
and can be propagated by the solver. 
So, deciding on true \code{move} atoms may be a good idea.
For that, we can either use the \code{true} modifier to express a strict preference
%\begin{lstlisting}[numbers=none]
\begin{lstlisting}[basicstyle=\small\ttfamily,numbers=none]
_heuristic(move(B,L,T),true,1) :- block(B),location(L),time(T).
\end{lstlisting}
or just bias the search with \code{init} and \code{sign}
%\begin{lstlisting}[numbers=none]
\begin{lstlisting}[basicstyle=\small\ttfamily,numbers=none]
_heuristic(move(B,L,T),init,2) :- block(B),location(L),time(T).
_heuristic(move(B,L,T),sign,1) :- block(B),location(L),time(T).
\end{lstlisting}
or with \code{factor} and \code{sign}
%\begin{lstlisting}[numbers=none]
\begin{lstlisting}[basicstyle=\small\ttfamily,numbers=none]
_heuristic(move(B,L,T),factor,2) :- block(B),location(L),time(T).
_heuristic(move(B,L,T),sign,  1) :- block(B),location(L),time(T).
\end{lstlisting}

So far, we have given the same heuristic values to all \code{move} atoms,
but other options may be interesting. 
For example, we may prefer to decide first on earlier \code{move} atoms,
so that the solver performs a forward search. 
This can be represented with the following rule:
%\begin{lstlisting}[numbers=none]
\begin{lstlisting}[basicstyle=\small\ttfamily,numbers=none]
_heuristic(move(B,L,T),true,lasttime-T+1) :- 
                             block(B), location(L), time(T).
\end{lstlisting}
For \code{lasttime=3},
the rule ranks \code{move} atoms at time \code{1} at level \code{3},
those at time \code{2} at level \code{2}, and 
those at time \code{3} at level \code{1},
while always assigning a positive \code{sign}.  
In this manner, 
the solver decides first on setting a \code{move} atom at time \code{1} to true, 
then one at time \code{2}, and so on.

Another strategy is to perform a backwards search on \code{move} from the last to the
first time instant, directed by the goals. 
For this purpose, we can use the following dynamic heuristic rule: 
%\begin{lstlisting}[numbers=none]
\begin{lstlisting}[basicstyle=\small\ttfamily,numbers=none]
_heuristic(move(B,L,T),true,T) :- on(B,L,T).
\end{lstlisting}
As before, the rule can be softened using \code{sign} with \code{init} or \code{factor}.
At the start of the search, the goal's \code{on} atoms are true at the last time step.
With this rule, the solver decides on a \code{move} atom that makes one of them true.
Then, some \code{on} atoms are propagated to the previous time step,
and the process is repeated until reaching the first time instant.

We can also choose to promote atoms of the \code{on} predicate.  
For example, this can be achieved with any of the  following rules.
%\begin{lstlisting}[numbers=none]
\begin{lstlisting}[basicstyle=\small\ttfamily,numbers=none]
_heuristic(on(B,L,T),level,1)  :- block(B),location(L),time(T).
_heuristic(on(B,L,T),init,2)   :- block(B),location(L),time(T).
_heuristic(on(B,L,T),factor,3) :- block(B),location(L),time(T).
\end{lstlisting}

Another interesting alternative is the following heuristic rule,  
that proved to be very useful in practice (see \cite{gekaotroscwa13a}):
%\begin{lstlisting}[numbers=none]
\begin{lstlisting}[basicstyle=\small\ttfamily,numbers=none]
_heuristic(on(B,L,T-1),true, lasttime-T+1) :- on(B,L,T).
\end{lstlisting}
The idea is to make the goal's \code{on} atoms persist backwards, 
one by one, from the last time step to the first one.
Note that a higher level is given to atoms at earlier time instants.
First, the solver decides on one of the goal's \code{on} atoms at the last but one time step,
then it decides to make it persist to the previous situation, and so on.
Later, it makes persist backwards another \code{on} atom from the goal.
With this heuristic,
the idea is not to decide on atoms that lead to much propagation (as with \code{move} atoms)
but rather to make correct decisions, 
given that usually the values of \code{on} atoms persist by inertia.
\end{example}


\subsection{Command Line Structure-oriented Heuristics}

The \code{Domain} heuristic also allows us to modify the heuristic of the solver from the command line.
For this, it is also activated with option \code{--heuristic=Domain}, 
but now the heuristic modifications are specified by option:
\begin{center}\lstinline{--dom-mod=<mod>[,<pick>]}\end{center}
where \code{<mod>} ranges from \code{0} to \code{5} and specifies the modifier and the value:
\begin{center}
\begin{tabular}{clc|clc}
\code{<mod>} & Modifier         & Value       & \code{<mod>} & Modifier     & Value\\
\hline 
\code{0}     & No   &                               & \code{1}     & \code{level} & \code{1}\\
\code{2}     & \code{sign}      & \code{ 1}                      & \code{3}     & \code{true}  & \code{1}\\
\code{4}     & \code{sign}      & \code{-1}                     & \code{5}     & \code{false} & \code{1}
\end{tabular}
\end{center}
and \code{<pick>} specifies bit-wisely the atoms to which the modification is applied:
\comment{JR2BK: Please check, I am not sure if that's right}
\begin{center}
\begin{tabular}{rcl}
\code{0}  & & All atoms\\
\code{1}  & & Atoms that belong to strongly connected components\\
\code{2}  & & Atoms that belong to head cycle free components\\
\code{4}  & & Atoms that appear in disjunctions\\
\code{8}  & & Atoms that appear in optimization statements\\
\code{16} & & Atoms that are shown
\end{tabular}
\end{center}
For example, with option \code{--dom-mod=5,16},
we apply modifier \code{false} with value \code{1} 
to atoms appearing in optimization statements, 
and with option \code{--dom-mod=1,20},
we apply modifier \code{level} with value \code{1}
to both atoms appearing in disjunctions as well as shown atoms.

Compared to programmed heuristics, 
the command line heuristics do not allow for applying modifiers \code{init} or \code{factor} 
and cannot represent dynamic heuristics.
On the other hand, they allow us to directly refer to structural components of the program 
(strongly connected components, head cycle free components, etc.)
and they do not require any additional grounding, 
which may be important in larger problems.
%
When both methods are combined,
the atoms modified by the \texttt{\_heuristic} predicate are not affected by the command line heuristics.

%%% Local Variables:
%%% mode: latex
%%% TeX-master: "guide"
%%% End:
