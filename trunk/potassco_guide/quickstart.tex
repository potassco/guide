\newcommand{\hanoiInstance}{
	\begin{tikzpicture}[->,>=stealth',semithick]
		% peg a 0.0 - 3.0
		\draw (0.5,0)   rectangle node {1} (2.5,0.5);
		\draw (0.7,0.5) rectangle node {2} (2.3,1);
		\draw (0.9,1)   rectangle node {3} (2.1,1.5);
		\draw (1.1,1.5) rectangle node {4} (1.9,2);
		\draw (1.4,2)   rectangle          (1.6,2.1);
		\draw [-] (1.5,2.1) node [above] {a};
		% peg b 3.0 - 6.0
		\draw (4.4,0.0)   rectangle          (4.6,2.1);
		\draw [-] (4.5,2.1) node [above] {b};
		% peg c 6.0 - 9.0
		\draw [dashed] (6.5,0)   rectangle node {1} (8.5,0.5);
		\draw [dashed] (6.7,0.5) rectangle node {2} (8.3,1);
		\draw [dashed] (6.9,1)   rectangle node {3} (8.1,1.5);
		\draw [dashed] (7.1,1.5) rectangle node {4} (7.9,2);
		\draw (7.4,2.0) rectangle          (7.6,2.1);
		\draw [-] (7.5,2.1) node [above] {c};
	\end{tikzpicture}
}%
\newcommand{\hanoiEncoding}{
	\begin{tikzpicture}[->,>=stealth',semithick]
		% peg a 0.0 - 3.0
		\draw<6> [color=alerted] (0.1,0)   rectangle node (Top)  {1} (2.9,0.5);
		\draw<4,7>               (0.1,0)   rectangle node (Top)  {1} (2.9,0.5);
		\draw                    (1.4,0.5) rectangle node (PegA) {}  (1.6,2.1);
		\draw [-] (1.5,2.1) node [above] {a};
		% peg b 3.0 - 6.0
		\draw<4,7> [color=alerted] (3.3,0)   rectangle node (Block) {2} (5.7,0.5);
		\draw<6>                   (3.3,0)   rectangle node (Block) {2} (5.7,0.5);
		\draw                      (4.3,0.5) rectangle node         {7} (4.7,1);
		\draw                      (4.4,1)   rectangle node (PegB)  {}  (4.6,2.1);
		\draw [-] (4.5,2.1) node [above] {b};
		% peg c 6.0 - 9.0
		\draw<4> [color=alerted] (6.5,0)   rectangle node {3} (8.5,0.5);
		\draw<6,7>               (6.5,0)   rectangle node {3} (8.5,0.5);
		\draw<4> [color=alerted] (6.7,0.5) rectangle node {4} (8.3,1);
		\draw<6,7>               (6.7,0.5) rectangle node {4} (8.3,1);
		\draw<4> [color=alerted] (6.9,1)   rectangle node {5} (8.1,1.5);
		\draw<6,7>               (6.9,1)   rectangle node {5} (8.1,1.5);
		\draw                    (7.1,1.5) rectangle node {6} (7.9,2);
		\draw                    (7.4,2)   rectangle          (7.6,2.1);
		\draw [-] (7.5,2.1) node [above] {c};
		\path<6> [bend left,color=alerted]  (Top)   edge node [-,draw,cross out] {} (PegB);
		\path<7> [bend right,color=alerted] (Block) edge node [-,draw,cross out] {} (PegA);
	\end{tikzpicture}
}%%
\section{Quickstart}\label{sec:quickstart}
In this section we demonstrate the expressive power 
and the simple yet powerful modeling language of \gringo{}
by looking at the simple Towers of Hanoi puzzle.
It consists of three pegs and a set of discs of different sizes, 
which can be put onto the pegs. 
The goal is to move all pegs from the leftmost peg to the rightmost peg,
where at each time only the topmost disc can be moved on top of another peg.
Additionally, a disc may not be put on top of a smaller disc.
We ignore that there is an efficient algorithm to solve this problem 
and just specify how a solution, in terms of a sequence of moves, has to look.

In ASP it is custom to provide a \emph{uniform}
problem definition~\cite{martru99a,niemela99a,schlipf95a}.
Following this methodology, we separate the encoding
from an instance of the following problem:
given an initial placement of the discs, a goal situation, and a number $n$,
decide whether there is a sequence of moves of length $n$ 
that satisfies the conditions given above.
We will see that this decision problem can be elegantly 
specified by reducing it to a declarative problem solving paradigm like ASP,
where efficient off-the-shelf tools like \gringo\ and \clasp\
are ready to solve the problem reasonably well.
Such a reduction is now exemplified.

\subsection{Problem Instance}

\begin{figure}[tb]
\centering
\hanoiInstance
\caption{Towers of Hanoi Initial Situation\label{fig:toh_inst}}
\end{figure}

We consider a Towers of Hanoi instance specified via facts over predicates
\pred{peg}/$1$ and \pred{disk}/$1$ that correspond to the pegs and disks in the puzzle.
Discs are enumerated by consecutive integers beginning with one,
where a disc with a lower number is considered to be bigger than a disc with a higher number.
The pegs can have arbitrary names. 
Furthermore, the predicates \pred{init\_on}/$2$ and \pred{goal\_on}/$2$ describe the initial and goal situation, respectively.
Their first argument is the number of a disc and the second argument is the peg 
on which the disc is located in the initial or goal situation.
Finally, the predicate \pred{moves}/$1$ specifies the number of moves within which the goal situation has to be reached.
Note that the original puzzle had exactly three pegs and a fixed initial and goal situation.
With ASP we can easily change this requirement and 
the encoding represented in the following works with an arbitrary number of pegs and any initial or goal situation.
Figure~\ref{fig:toh_inst} depicts a possible instance (the dashed discs mark the goal situation) 
corresponding to the ASP program given below.

\begin{lstlisting}
peg(a;b;c).
disk(1..4).
init_on(1..4,a).
goal_on(1..4,c).
moves(15).
\end{lstlisting}

The ``\code{;}'' in the first line is some syntactic sugar (Section \ref{subsec:gringo:pool})
that expands the statement into three facts
\code{peg(a)}, \code{peg(b)}, and \code{peg(c)} representing the three pegs.
Again, in the second line some syntactic sugar is used to 
create the facts \code{disc(1)}, \code{disc(2)}, \code{disc(3)}, and \code{disc(4)}.
Here the term \code{1..4}, an intervall (Section \ref{subsec:gringo:interval}), is successively replaced by \code{1}, \code{2}, \code{3}, and \code{4}.
The initial and goal situation is specified in line three and four again using intervall.
Finally, in the last line the number of moves to solve the problem is given.

\subsection{Problem Encoding}

We now proceed by encoding the Towers of Hanoi puzzle via non-ground rules (Section \ref{subsec:gringo:normal}), 
i.e, rules with variables that are independent of particular instances.
Typically, an encoding consists of a \emph{Generate}, a \emph{Define},
and a \emph{Test} part~\cite{lifschitz02a}.
We follow this paradigm and
mark respective parts via comment lines beginning with \code{\%} in the encoding below.
The variables \code{D}, \code{P}, \code{T}, and \code{M} are used
to refer to disks, pegs, the \code{T}-th move in the sequence of moves, and the length of the sequence, respectively.

\begin{lstlisting}
% Generate
1 { move(D,P,T) : disk(D) : peg(P) } 1 :- moves(M), T = 1..M.
% Define
move(D,T)   :- move(D,_,T).
on(D,P,0)   :- init_on(D,P).
on(D,P,T)   :- move(D,P,T).
on(D,P,T+1) :- on(D,P,T), not move(D,T+1), not moves(T).
blocked(D-1,P,T+1) :- on(D,P,T), disk(D), not moves(T).
blocked(D-1,P,T)   :- blocked(D,P,T), disk(D).
% Test
:- move(D,P,T), blocked(D-1,P,T).
:- move(D,T), on(D,P,T-1), blocked(D,P,T).
:- goal_on(D,P), not on(D,P,M), moves(M). 
:- not 1 { on(D,P,T) : peg(P) } 1, disk(D), moves(M), T = 1..M.
#hide.
#show move/3. 
\end{lstlisting}%
The Generate part consists of just one rule in Line 2.
At each time point \code{T} at which a move is executed, we
``guess'' exactly one move that puts an arbitrary disk to some arbitrary peg.
The head of this rule is a so called cardinality constraint (Section \ref{subsec:gringo:aggregate}) that
consists of a set that is expanded using the predicates behind the colons (Section \ref{subsec:gringo:condition}) and a lower and an upper bounds.
The constraint is true if and only if the number of true literals within the set is between the upper and lower bound.
Furthermore, the constraint is used in the head of a rule, 
that is, it is not only a test but can derive (``guess'') new atoms, which in this case correspond to possible moves of discs.
Note that at this point we have not constrained the moves.
Up to now, any disc could be moved to any peg at each time point with out considering any problem constraints.

Next follows the Define part, here we give rules that define new auxiliary predicates, 
which as such do not touch the satisfiability of the problem but are used in the Test part later on.
The rule in Line 4 projects out the target peg of a move, i.e., the predicate \code{move}/$2$ can be used 
if we only need the disc affected by a move but not its target location.
We use the predicate \code{on}/$3$ to capture the state of the Hanoi puzzle at each time point.
Its first two argument give the location of a disc at the time point given by the third argument.
The next rule in Line 5 infers the location of each disc in the initial state (time point 0).
Then we model the state transition using the rules in Line 6 and 7.
The first rule is quite straightforward and states that the moved disc changes its location.
Note the usage of \code{not moves(T)} here.
This literal prevents deriving an infinite number of rules, which would be all useless 
because the state no longer changes after the last move.
The second rule makes sure that all discs that are not moved stay where they are.
Finally, we define the auxiliary predicate \code{blocked}/$3$, 
which marks positions w.r.t. pegs that cannot be moved from.
First in Line 8, the position below a disc on some peg is blocked.
Second in Line 9, the position directly below a blocked position is blocked.
Note that we mark position zero to be blocked, too.
This is convenient later on to assert some redundant moves.

Finally, there is the Test part building upon both Generate and Define part 
to rule out wrong moves that do not agree with the problem description.
It consists solely of integrity constraints, which fail whenever all their literals are true.
The first integrity constraint in Line 11 asserts 
that a disc that is blocked, i.e, with some disc on top, cannot be moved.
Note the usage of \code{D-1} here, this way a disc cannot be put back to the same location again.
The integrity constraint in Line 12 asserts that a disc can only be placed on top of a bigger disc.
Line 13 asserts the goal situation.
To make the encoding more efficient, we add a redundant constraint in Line 14,
which asserts that each disc at all time points is located on exactly one peg.
Although, this constraint is implied by the constraints above,
adding this additional domain knowledge greatly improves the speed with which the problem can be solved.
Finally, the last two statements control which predicates are printed, when a satisfying model
for the instance is found.
Here we first hide all predicates (Line 15) and then explicitly show only the \code{move}/$3$ predicate (Line 16).

\subsection{Problem Solution}

Now we are ready to solve the encoded puzzle. 
To find an answer set, invoke one of the following commands 
(\clingo{}, or \gringo{} and \clasp{} have to be installed somewhere under the systems path for the commands below to work):
\begin{lstlisting}[numbers=none,escapechar=\%]
clingo %\attach{examples/inst_toh.lp}{inst\char`\_toh.lp}% %\attach{examples/enc_toh.lp}{enc\char`\_toh.lp}%
gringo %\attach{examples/inst_toh.lp}{inst\char`\_toh.lp}% %\attach{examples/enc_toh.lp}{enc\char`\_toh.lp}% | clasp
\end{lstlisting}
Note that (depending on your viewer) you can right or double-click on file names marked with a red font 
to safe the associated file to disc.
This is possible with all examples given in this document.

The output of the solver (\clingo{} in this case) looks something like that:
\begin{lstlisting}[numbers=none]
Answer: 1
move(4,b,1)  move(3,c,2)  move(4,c,3)  move(2,b,4)  \
move(4,a,5)  move(3,b,6)  move(4,b,7)  move(1,c,8)  \
move(4,c,9)  move(3,a,10) move(4,a,11) move(2,c,12) \
move(4,b,13) move(3,c,14) move(4,c,15) 
SATISFIABLE

Models      : 1+    
Time        : 0.010
  Prepare   : 0.000
  Prepro.   : 0.010
  Solving   : 0.000
\end{lstlisting}
The first line indicates that an answer set follows in the line below (the \code{\textbackslash} marks a line wrap).
Then the status follows, this might be either \code{SATISFIABLE}, \code{UNSATISFIABLE}, or \code{UNKNOW} 
if the computation is interrupted.
The \code{1+} right of \code{Models:} indicates that one answer set has been found and the \code{+} that 
the whole search space has not yet been explored, so there might be further answer sets.
Following that, there are some time measurements: 
Beginning with total computation time, 
which is split into 
preparation time (grounding), 
preprocessing time (\clasp{} has an internal preprocessor that tries to simplify the program), 
and solving time (the time needed to find the answer excluding preparation and preprocessing time).
More information about options and output can be found in Section~\ref{sec:options}.
