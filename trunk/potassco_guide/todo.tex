\section{TODO}
\com{TODO}
\begin{itemize}
% \item clingo
%   \begin{itemize}
%   \item Python/Lua
%   \item etc
%   \end{itemize}
% \item domains heuristic
%   \begin{itemize}
%   \item options
%   \item heuristic
%   \end{itemize}
% \item meta-programming
% \item CSP \$
% \item abstract for laymen
% \item more prominent mention of AG (large subset of syntex and semantics precisely described)
% \item remark and example environment
% \item use comparison predicates in aggregates rather than identical lower and upper bounds
\item restructure index (lowercase, structure see lp)
%%% \item section disjunctive modeling (do book)
% \item quoting of expressions/symbols (double quotes, punctuation)
\item homogeneous style for schematics in language section
\item the language section uses very often: ``gringo and the grounding component of clingo''
\item Vladimir's comments
\begin{itemize}
 \item 
  The symbols \code{\#sup} and \code{\#inf} are new to me, is this a recent addition to the
  language?  Is it really true that \code{\#sup>f(\#sup)}, but \code{\#inf<f(\#inf)}?
  The explanation of these symbols at the bottom of page 13 is cryptic, I’m
  afraid, unless you say there that a total order on variable-free terms is
  going to be inroduced later.
 \item
  It seems that you banished the comparison symbol ==, and I wholeheartedly
  approve this decision.  But then you need to include = in the list of
  comparison symbols, instead of relegating it to the section on assignments,
  isn't that right? Otherwise X+Y=Z*U will not be syntactically correct. 
  Also, the examples in the middle of page 27, with “=2” in the body, are
  syntactically incorrect if equalities are not included among comparison
  symbols.
 \item
  The word ``assignment'' is used many times in the draft.  My advice to Roland
  (which, as, he found reasonable) was to refrain from using it, because it
  sounds terribly nondeclarative, and because “assignments” are not a special
  syntactic category now that we don't have == . Instead you can say that
  comparisons of the form A=B where A or B is a variable are processed in a
  special way so that the variable becomes safe.  Or something like this. 
  Similarly with ``assignment aggregates.''
 \item
  The discussion of terms in Sec. 3.1.1 gives the impression that Fig. 2 is a
  complete description of the syntax of terms.  It would be good to say here
  that the definition of a term will be extended later, when arithmetic
  operations and intervals are introduced.  In fact, Fig. 2 defines something
  close to what we call “precomputed” terms in the AG paper.  It may be
  worthwhile to include this (or similar) name for the class of terms covered
  by Fig. 2, for the following reason.  The total order that you talk about
  in Sec. 3.1.7 is not defined actually on all variable-free terms; it is
  defined on *precomputed* variable free terms.  Once we decided whether f(a)
  is greater than g(2), we are committed to the same choice regarding f(a)
  and g(1+1), and regarding g(1..1), right?
 % \item
 %  Will the reader understand “cannot span positive cycles” and “induces no
 %  positive cycle” in Sec. 3.1.4?  Unfortunately, I don't know what to
 %  suggest.
  \end{itemize}
\item Christoph's comments
  \begin{itemize}
  \item 
    I have one comment regarding the future work section where it says that it
    is considered to add "support for arbitrary positive loops". Since the
    second half of the sentence talks about redefining atoms in incremental
    programs, I was wondering if these two features are meant to be used
    together, i.e., redefining atoms in a cyclic fashion (which would
    contradict the outcome of our discussion after Cristina's defense). If not,
    then it should be clarified which kind of positive loops are meant here
    since most positive loops are already supported (maybe over aggregates?).
  \item 
    Another question concerns Section 3.1.11 (conditional literals), where I
    was wondering how the rule would be instantiated if person(jane) and
    person(john) were no fact but derivable atoms. Then meet would only depend
    on the available atoms whose corresponding person atoms are currently true.
    Maybe one should give another example which demonstrates this. (My idea
    would be to use default-negation to derive an intermediate atom if there is
    a person who is not available, and then use another default-negation to
    check if this atom is not true.).
\end{itemize}
\end{itemize}

%%% Local Variables: 
%%% mode: latex
%%% TeX-master: "guide"
%%% End: 
